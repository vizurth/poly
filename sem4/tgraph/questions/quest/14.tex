{
	\section{Эвристические алгоритмы. Алгоритм А*. Метод ветвей и границ}

\subsection*{Эвристика}
Эвристикой называется алгоритм решения задачи, включающий практический метод, не являющийся гарантированно точным или оптимальным, но достаточный для получения решения и позволяющий ускорить процесс вычислений.

\begin{itemize}
  \item Не гарантирует лучшее решение.
  \item Не гарантирует наличие решения.
  \item Может дать неверное решение.
\end{itemize}

\subsection*{Алгоритм A*}

Порядок обхода вершин определяется \textbf{эвристической функцией} $f(x)$, которая представляет собой сумму двух компонент:


\[
f(x) = g(x) + h(x),
\]


где:
\begin{itemize}
  \item $g(x)$ — функция стоимости достижения вершины $x$ из начальной вершины (может быть эвристической или точной);
  \item $h(x)$ — эвристическая оценка расстояния от вершины $x$ до целевой вершины.
\end{itemize}

Функция $h(x)$ должна быть \textbf{допустимой эвристикой}, то есть не должна переоценивать расстояние до цели.  
Например, в задаче маршрутизации $h(x)$ может представлять собой расстояние до цели по прямой линии.

\subsection*{Интуиция}
На каждом шаге алгоритм оценивает, насколько текущая вершина приближает к цели, используя функцию $f(x)$, и выбирает путь, минимизирующий эту оценку.

Ну короче типо мы смотрим на каждом шаге насколько мы вообще
приблизились к цели или отдалились с помощью какой-либо
эвристической функции и это учитываем при выборе пути

\section*{Метод ветвей и границ}

Метод ветвей и границ является развитием метода полного перебора, но с отсевом подмножеств, заведомо не содержащих оптимальных решений.

\subsection*{Идея}
На каждом шаге элементы разбиения анализируются: содержит ли подмножество оптимальное решение или нет.  
Если решается задача минимизации, то проверка осуществляется сравнением нижней оценки значения целевой функции с верхней оценкой функционала.

\subsection*{Рекорд}
Допустимое решение, дающее наименьшую верхнюю оценку, называется \textbf{рекордом}.  
Если нижняя оценка целевой функции на данном подмножестве не меньше текущего рекорда, то это подмножество не содержит лучшего решения и может быть отброшено.  
Если значение целевой функции меньше рекорда, рекорд обновляется.

\subsection*{Завершение}
Если все элементы разбиения просмотрены, алгоритм завершает работу, и текущий рекорд является оптимальным решением.  
Если нет — выбирается перспективное множество, которое подвергается дальнейшему разбиению.  
Процесс продолжается, пока не будут просмотрены все элементы.


    \newpage
}
