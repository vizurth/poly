{
	\section{Теорема Менгера (с доказательством). Непересекающиеся цепи и разделяющие
множества. Варианты теоремы Менгера. Теорема Холла.}

\section*{Теорема Менгера}

Пусть $u$ и $v$ — несмежные вершины в графе $G$.  
Наименьшее число вершин в множестве, разделяющем $u$ и $v$, равно наибольшему числу вершинно непересекающихся простых цепей $\langle u, v \rangle$:



\[
\max |P(u, v)| = \min |S(u, v)|
\]



где:
\begin{itemize}
  \item $P(u, v)$ — множество вершинно непересекающихся цепей $\langle u, v \rangle$;
  \item $S(u, v)$ — разделяющее множество вершин (удаление которых приводит к разбиению графа так, что $u$ и $v$ оказываются в разных компонентах связности).
\end{itemize}

я ебал это доказывать

\subsection*{Разрез}
Разделяющее множество рёбер называется \textbf{разрезом}.

\section*{Варианты теоремы Менгера}

\subsection*{Теорема}
Для любых двух несмежных вершин $u$ и $v$ графа $G$ наибольшее число рёберно-непересекающихся цепей $\langle u, v \rangle$ равно наименьшему числу рёбер в $\langle u, v \rangle$-разрезе.

\subsection*{Теорема}
Граф $G$ является $n$-связным тогда и только тогда, когда любые две несмежные вершины соединены не менее чем $n$ вершинами, образующими вершинно-непересекающиеся простые цепи.

\section*{Теорема Холла}

\subsection*{Паросочетание}
\textbf{Паросочетанием} (или независимым множеством рёбер) называется множество рёбер, никакие два из которых не смежны.  
Паросочетание называется \textbf{максимальным}, если никакое его надмножество не является независимым.

Пусть $G(V_1, V_2, E)$ — двудольный граф.  
\textbf{Совершенным паросочетанием} из $V_1$ в $V_2$ называется паросочетание, покрывающее все вершины множества $V_1$.

\subsection*{Теорема Холла}
Совершенное паросочетание существует тогда и только тогда, когда


\[
\forall A \subseteq V_1 \quad (|A| \leq |\Gamma(A)|),
\]



    \newpage
}
