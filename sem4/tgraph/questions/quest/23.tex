{
	\section{Информационные деревья. А- и B-деревья. Красно-черные деревья.}

	\section*{A-дерево}

\textbf{A-дерево} — это разновидность сбалансированного дерева поиска, разработанная как альтернатива B-дереву для эффективной работы с большими объёмами данных в системах хранения.

\subsection*{Основные особенности}

\begin{itemize}
  \item A-дерево оптимизировано для хранения на внешних носителях (например, дисках), минимизируя количество операций ввода-вывода.
  \item В отличие от B-дерева, A-дерево использует \textbf{буферизацию} в узлах, позволяя накапливать операции и выполнять их пакетно.
  \item Каждый узел содержит буфер, в котором временно хранятся вставки, удаления и обновления, прежде чем они будут распространены вниз по дереву.
  \item Это позволяет значительно сократить количество обращений к диску при массовых операциях.
\end{itemize}

\subsection*{Структура узла}

Каждый узел A-дерева содержит:
\begin{itemize}
  \item Упорядоченный список ключей.
  \item Буфер операций (вставка, удаление, обновление).
  \item Указатели на дочерние узлы.
\end{itemize}

\subsection*{Преимущества}

\begin{itemize}
  \item Эффективность при пакетной обработке данных.
  \item Подходит для систем с ограниченным доступом к памяти и высокой стоимостью операций ввода-вывода.
  \item Хорошо масштабируется при работе с большими индексами.
\end{itemize}

\subsection*{Недостатки}

\begin{itemize}
  \item Более сложная реализация по сравнению с классическим B-деревом.
  \item Задержка при выполнении отдельных операций из-за буферизации.
\end{itemize}



	\section*{B-дерево}

\subsection*{Общие сведения}

B-дерево — это сбалансированное, сильно ветвистое дерево, широко применяемое для организации индексов в системах управления базами данных (СУБД).

\subsection*{Разновидности}

\begin{itemize}
  \item \textbf{B-дерево:} данные хранятся в любом узле.
  \item \textbf{B*-дерево:} данные хранятся в любом узле, но узлы заполняются на две трети.
  \item \textbf{B+-дерево:} данные хранятся только в листьях, во внутренних узлах — копии ключей.
\end{itemize}

\subsection*{Свойства B-дерева}

B-деревом называется дерево, удовлетворяющее следующим условиям:

\begin{enumerate}
  \item Ключи в каждом узле упорядочены для быстрого доступа.  
  Корень содержит от $1$ до $2t - 1$ ключей.  
  Любой другой узел содержит от $t - 1$ до $2t - 1$ ключей.  
  Листья не являются исключением.  
  Здесь $t \ge 2$ — параметр дерева.

  \item Листья не имеют потомков.  
  Узел с ключами $K_1, \ldots, K_n$ имеет $n + 1$ потомков.  
  Потомки содержат ключи, попадающие в интервалы между ключами узла.  
  Иначе говоря, каждый внутренний узел можно представить как упорядоченный список, в котором чередуются ключи и указатели на потомков.

  \item Все листья находятся на одном уровне, то есть имеют одинаковую глубину.
\end{enumerate}



	\section*{Красно-чёрное дерево}

\textbf{Красно-чёрным деревом} называется бинарное поисковое дерево, в котором каждому узлу сопоставлен дополнительный атрибут — цвет (красный или чёрный), и выполняются следующие свойства:

\begin{enumerate}
  \item Каждый узел промаркирован красным или чёрным цветом.

  \item Корень дерева и все листья (конечные узлы) — чёрные.

  \item У красного узла родитель обязательно чёрный.

  \item Все простые пути от любого узла $x$ до листьев содержат одинаковое количество чёрных узлов.

  \item Чёрный узел может иметь чёрного родителя.
\end{enumerate}


    \newpage
}
