{
	\section{Построение независимых множеств вершин. Поиск с возвратами. Улучшенный
перебор. Доминирующие множества. Доминирование и независимость. Задача о
наименьшем покрытии.}

\section*{Построение независимых множеств вершин. Поиск с возвратами}

Пусть задан граф $G(V, E)$. Требуется найти такое множество вершин $X \subset V$, что:


\[
w(X) = \max_{Y \in \mathcal{E}} w(Y)
\]


где:


\[
w(Y) \overset{\text{def}}{=} |Y|,\quad \mathcal{E} \overset{\text{def}}{=} \left\{ Y \subset V \mid \forall u, v \in Y,\ (u, v) \notin E \right\}
\]



То есть $X$ — максимальное по мощности независимое множество вершин.

\subsection*{Методы решения}

\begin{itemize}
  \item Полный перебор всех подмножеств $2^V$ с проверкой условия независимости.
  \item Поиск с возвратами (backtracking) — перебор с отсечениями, позволяющий избежать рассмотрения заведомо неподходящих конфигураций.
\end{itemize}

\begin{figure}[h]
\centering
\includegraphics[width=0.8\textwidth]{Photo/algo_30.jpg}
% \caption{Списки смежности для графа $G$ и орграфа $D$}
\end{figure}

\section*{Улучшенный перебор}

Идея: начинаем с пустого множества и пополняем его вершинами с сохранением независимости (пока возможно).  
За подробностями пойдёт нахуй.

\section*{Доминирующие множества}

Множество вершин $S$ графа $G(V, E)$ называется \textbf{доминирующим}, если:


\[
S \cup \Gamma(S) = V
\]


Очевидно, что множество $S$ доминирует тогда и только тогда, когда:


\[
\forall v \notin S,\quad \Gamma(v) \cap S \neq \emptyset
\]



Доминирующее множество называется:
\begin{itemize}
  \item \textbf{Минимальным}, если никакое его подмножество не является доминирующим;
  \item \textbf{Наименьшим}, если его мощность минимальна среди всех доминирующих множеств.
\end{itemize}

\section*{Доминирование и независимость}

\textbf{Теорема.} Независимое множество вершин является \textbf{максимальным} тогда и только тогда, когда оно является \textbf{доминирующим}.

\section*{Задача о наименьшем покрытии}

Каждой вершине сопоставлена некоторая цена.  
Требуется выбрать доминирующее множество с наименьшей суммарной ценой — это \textbf{задача о наименьшем покрытии}.

Для её решения применяются переборные алгоритмы с теми или иными улучшениями.



    \newpage
}
