{
	\section{Гиперграфы. Двойственные гиперграфы. Циклы и реализации.}

	\section*{Гиперграфы. Двойственные гиперграфы}

\subsection*{Определение гиперграфа}

Гиперграф — это обобщение графа, в котором рёбра могут соединять произвольные подмножества вершин, а не только пары.

Пусть:
\begin{itemize}
  \item $V$ — конечное непустое множество вершин;
  \item $E$ — семейство непустых подмножеств множества $V$;
\end{itemize}

Тогда пара $(V, E)$ называется \textbf{гиперграфом}.

Обозначения:
\begin{itemize}
  \item $VH$ — множество вершин гиперграфа $H$;
  \item $EH$ — множество рёбер гиперграфа $H$;
  \item $|H| = n$ — число вершин;
  \item $|EH| = m$ — число рёбер;
  \item $H$ называется $(n, m)$-гиперграфом.
\end{itemize}

\subsection*{Двойственный гиперграф}

Пусть $H(V, E)$ — гиперграф без изолированных вершин.  
Его \textbf{двойственным гиперграфом} называется гиперграф $H^*(V^*, E^*)$, где:
\begin{itemize}
  \item $V^* = E$;
  \item $E^* = \{E(v) \mid v \in V\}$ — семейство множеств рёбер, содержащих вершину $v$.
\end{itemize}

\section*{Циклы и реализация}

\subsection*{Циклы в гиперграфе}

Гиперграф не содержит циклов тогда и только тогда, когда для любого непустого подмножества $V' \subseteq V$ выполнено:


\[
\left| \bigcup_{v \in V'} \delta(v) \right| > \sum_{v \in V'} (\delta(v) - 1)
\]


где $\delta(v)$ — множество рёбер, содержащих вершину $v$.

\subsection*{Реализация гиперграфа}

Граф $G$ называется \textbf{реализацией} гиперграфа $H(V, E)$, если выполняются следующие условия:
\begin{enumerate}
  \item $VG = VH$;
  \item Каждое ребро графа $G$ содержится в некотором ребре гиперграфа $H$;
  \item Для любого ребра $e \in E$ подграф $G_e$ является связным и $VG_e = e$.
\end{enumerate}

Любая реализация гиперграфа является объединением реализаций его рёбер.


    \newpage
}
