{
	\section{Представление деревьев в ЭВМ. Обходы бинарных деревьев. Алгоритм
симметричного обхода бинарного дерева}

\section*{Представление бинарных деревьев в ЭВМ}

Существует несколько способов представления бинарных деревьев в памяти ЭВМ:

\subsection*{1. Списочные структуры}

Каждый узел представляется записью типа $N$, содержащей:
\begin{itemize}
  \item поле $i$ — информация, связанная с узлом;
  \item указатели $l$ и $r$ — на левый и правый дочерние узлы соответственно.
\end{itemize}

Тип данных:


\[
N = \text{record } i : \text{info};\ l, r : \uparrow N\ \text{end record}
\]



Дерево представляется указателем на корень.  
Объём памяти:


\[
n(p) = 3p
\]



\subsection*{2. Упакованные массивы}

Все узлы размещаются в массиве так, чтобы узлы поддерева следовали за текущим узлом.  
Каждый узел хранит индекс первого узла правого поддерева.

Тип данных:


\[
T : \text{array } [1..p] \text{ of record } i : \text{info},\ k : \uparrow\ \text{end record}
\]



Объём памяти:


\[
n(p) = 2p
\]



\subsection*{3. Польская запись}

Вместо указателей фиксируется «степень» узла:
\begin{itemize}
  \item $0$ — лист;
  \item $1$ — только левый потомок;
  \item $2$ — только правый потомок;
  \item $3$ — оба потомка.
\end{itemize}

Тип данных:


\[
T : \text{array } [1..p] \text{ of record } i : \text{info},\ d : 0..3\ \text{end record}
\]



Объём памяти:


\[
n(p) = 2p
\]



\section*{Обходы бинарных деревьев}

Существует три основных способа обхода бинарного дерева:

\subsection*{1. Прямой обход (префиксный, левый) preorder}
\begin{itemize}
  \item Посетить корень.
  \item Обойти левое поддерево.
  \item Обойти правое поддерево.
\end{itemize}

\subsection*{2. Внутренний обход (инфиксный, симметричный) inorder}
\begin{itemize}
  \item Обойти левое поддерево.
  \item Посетить корень.
  \item Обойти правое поддерево.
\end{itemize}

\subsection*{3. Концевой обход (постфиксный, правый) postorder}
\begin{itemize}
  \item Обойти левое поддерево.
  \item Обойти правое поддерево.
  \item Посетить корень.
\end{itemize}

Кроме этих трёх, возможны ещё три варианта обхода, отличающиеся порядком рассмотрения левого и правого поддеревьев.  
Все возможные обходы исчерпываются, если в представлении фиксированы только дуги, ведущие от отцов к сыновьям.

\section*{Алгоритм симметричного обхода бинарного
дерева}

\begin{figure}[h]
\centering
\includegraphics[width=0.8\textwidth]{Photo/algo_20.jpg}
% \caption{Списки смежности для графа $G$ и орграфа $D$}
\end{figure}



    \newpage
}
