{
	\section{Ориентированные, упорядоченные и бинарные деревья. Свойства ордерева.
Эквивалентное определение ордерева. Упорядоченные деревья.}

\section*{Ориентированные, упорядоченные и бинарные деревья. Свойства ор-
дерева}

Ориентированным деревом (или \textbf{ордеревом}, или \textbf{корневым деревом}) называется орграф $G = (V, E)$, обладающий следующими свойствами:

\begin{enumerate}
  \item Существует единственный узел $r$, полустепень захода которого равна нулю:  
  

\[
  d^+(r) = 0.
  \]


  Узел $r$ называется \textbf{корнем} ордерева.

  \item Полустепень захода всех остальных узлов равна единице:  
  

\[
  \forall v \in V \setminus \{r\},\quad d^+(v) = 1.
  \]



  \item Каждый узел достижим из корня:  
  

\[
  \forall v \in V \setminus \{r\},\quad \exists\ \text{ориентированный путь } (r \leadsto v).
  \]


\end{enumerate}

\subsection*{Теорема}

Ориентированное дерево обладает следующими свойствами:

\begin{enumerate}
  \item Число рёбер:  
  

\[
  q = p - 1.
  \]



  \item Если забыть ориентацию дуг, то получится свободное (неориентированное) дерево.

  \item В ордереве нет контуров.

  \item Для каждого узла существует единственный путь от корня к этому узлу.

  \item Подграф, состоящий из узлов, достижимых из узла $v$, является ордеревом с корнем $v$ (такое ордерево называется \textbf{поддеревом} узла $v$).

  \item Если в свободном дереве любую вершину назначить корнем, то получится ордерево.
\end{enumerate}

\section*{Эквивалентное определение ордерева.
Упорядоченные деревья}

Ордерево $T$ — это непустое конечное множество узлов, на котором задано разбиение, обладающее следующими свойствами:

\begin{enumerate}
  \item Существует один выделенный одноэлементный блок $\{r\}$, называемый \textbf{корнем} данного ордерева.

  \item Остальные узлы (исключая корень) содержатся в $k$ блоках $T_1, T_2, \ldots, T_k$, где $k \ge 0$.  
  Каждый из блоков $T_i$ является ордеревом и называется \textbf{поддеревом}.
\end{enumerate}

Таким образом, ордерево можно записать как:


\[
T \overset{\text{def}}{=} \{\{r\}, T_1, T_2, \ldots, T_k\}.
\]



Если относительный порядок поддеревьев $T_1, T_2, \ldots, T_k$ фиксирован, то ордерево называется \textbf{упорядоченным}.


    \newpage
}
