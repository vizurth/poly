{
	\section{Связность в орграфах (сильная, односторонняя и слабая связность, компоненты
сильной связности). Алгоритм выделения компонент сильной связности.}

\section*{Связность в ориентированных графах}

В неориентированном графе две вершины либо связаны (если существует соединяющая их цепь), либо не связаны.  
В ориентированном графе отношение связности узлов несимметрично, поэтому определения связности различаются.

Пусть $G(V, E)$ — орграф, $v_1$ и $v_2$ — его вершины.

\begin{itemize}
  \item \textbf{Сильная связность:}  
  Вершины $v_1$ и $v_2$ \textbf{сильно связаны} в орграфе $G$, если существует путь из $v_1$ в $v_2$ и путь из $v_2$ в $v_1$.

  \item \textbf{Односторонняя связность:}  
  Вершины $v_1$ и $v_2$ \textbf{односторонне связаны}, если существует путь либо из $v_1$ в $v_2$, либо из $v_2$ в $v_1$.

  \item \textbf{Слабая связность:}  
  Вершины $v_1$ и $v_2$ \textbf{слабо связаны}, если они связаны в неориентированном графе $G'$, полученном из $G$ забыванием ориентации дуг.
\end{itemize}

Если все вершины орграфа $G$ сильно (односторонне, слабо) связаны между собой, то орграф называется соответственно \textbf{сильно}, \textbf{односторонне} или \textbf{слабо связным}.

\subsection*{Иерархия связности}
Сильная связность $\Rightarrow$ односторонняя связность $\Rightarrow$ слабая связность.  
Обратные импликации, как правило, неверны.


\subsection*{Компонента сильной связности}

\textbf{Компонентой сильной связности} (КСС) орграфа $G$ называется его максимальный сильно связный подграф.  
Каждая вершина орграфа принадлежит ровно одной КСС.  
Если вершина не сильно связана ни с одной другой, она сама образует КСС.

\subsection*{Конденсация орграфа}

\textbf{Конденсацией} орграфа $G$ (обозначается $G^*$), также называемой \textbf{графом Герца} или \textbf{фактор-графом}, называется орграф, полученный путём стягивания каждой компоненты сильной связности в один узел.

В результате конденсации получается ациклический орграф, отражающий структуру связности исходного графа на уровне компонент.

\begin{figure}[h]
\centering
\includegraphics[width=0.8\textwidth]{Photo/ALGO_17.jpg}
% \caption{Списки смежности для графа $G$ и орграфа $D$}
\end{figure}

\subsection*{Рекурсивная процедура KCC}

Основная работа выполняется рекурсивной процедурой \texttt{KCC}, не принимающей параметров.  
Процедура использует стек $T$ для хранения просматриваемых узлов и выделяет все компоненты сильной связности, достижимые из узла, выбранного в основном алгоритме.

\subsubsection*{Интуиция}
Любой контур принадлежит ровно одной компоненте сильной связности (КСС).  
Если КСС содержит несколько узлов, то они обязательно входят в один или несколько контуров.

\subsubsection*{Обработка при обходе}
Если при обходе в глубину мы попадаем в уже отмеченный узел $w$, это означает, что обнаружен контур.  
Предшествующие узлы этого контура находятся на стеке — от вершины стека до узла $w$, который также присутствует в стеке.

\subsubsection*{Склеивание}
Все узлы найденного контура можно «склеить»:
\begin{itemize}
  \item Список смежности узла $w$ объединяется со списком смежности узла $i$.
  \item Список уже найденных узлов КСС, к которой принадлежит $w$, объединяется с соответствующим списком узла $i$.
\end{itemize}

После склеивания поиск в глубину продолжается от узла $i$, который остаётся в стеке.



    \newpage
}
