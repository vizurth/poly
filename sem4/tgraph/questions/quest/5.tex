{
	\section{Способы задания графа. Представление графов в ЭВМ. Обходы графов.}

	\subsection*{Представление графов в программах}

	Следует ещё раз подчеркнуть, что конструирование структур данных для представления в программе объектов математической модели — это основа искусства практического программирования.

	Мы приводим четыре различных базовых представления графов. Выбор наилучшего представления определяется требованиями конкретной задачи. Более того, на практике используются, как правило, некоторые комбинации или модификации указанных представлений, общее число которых необозримо. Но все они так или иначе основаны на тех базовых идеях, которые описаны в этом разделе.

	\subsection*{Требования к представлению графов}

	Известны различные способы представления графов в памяти компьютера, которые различаются объёмом занимаемой памяти и скоростью выполнения операций над графами. Представление выбирается, исходя из потребностей конкретной задачи.

	Далее приведены четыре наиболее часто используемых представления с указанием характеристики $\eta(p, q)$ — объёма памяти для каждого представления, где $p$ — число вершин, а $q$ — число рёбер.
	
	\subsection*{7.4.2. Матрица смежности}

	Представление графа с помощью квадратной булевой матрицы


	\[
	M : \text{array } [1..p, 1..p] \text{ of } \{0, 1\},
	\]


	отражающей смежность вершин, называется \textbf{матрицей смежности}, где


	\[
	M[i, j] =
	\begin{cases}
	1, & \text{если вершина } v_i \text{ смежна с вершиной } v_j, \\
	0, & \text{если вершины } v_i \text{ и } v_j \text{ не смежны}.
	\end{cases}
	\]



	Для матрицы смежности объём памяти:


	\[
	\eta(p, q) = \mathcal{O}(p^2).
	\]



	\paragraph{Пример.} Матрицы смежности графов $G$ и $D$:


	\[
	G =
	\begin{pmatrix}
	0 & 1 & 0 & 1 \\
	1 & 0 & 1 & 1 \\
	0 & 1 & 0 & 1 \\
	1 & 1 & 0 & 0
	\end{pmatrix}
	\qquad
	D =
	\begin{pmatrix}
	0 & 1 & 0 & 0 \\
	0 & 0 & 1 & 1 \\
	0 & 0 & 0 & 0 \\
	1 & 0 & 1 & 0
	\end{pmatrix}
	\]

	\paragraph{Замечание.}
Матрица смежности графа симметрична относительно главной диагонали, поэтому достаточно хранить только верхнюю (или нижнюю) треугольную часть.

\subsection*{Матрица инцидентций}

Представление графа с помощью матрицы


\[
H : \text{array } [1..p, 1..q] \text{ of } \{0, 1\},
\]


а для орграфов:


\[
H : \text{array } [1..p, 1..q] \text{ of } \{-1, 0, 1\},
\]


отражающей инцидентность вершин и рёбер, называется \textbf{матрицей инцидентций}, где для неориентированного графа:


\[
H[i, j] =
\begin{cases}
1, & \text{если вершина } v_i \text{ инцидентна ребру } e_j, \\
0, & \text{в противном случае}.
\end{cases}
\]



А для ориентированного графа:


\[
H[i, j] =
\begin{cases}
1, & \text{если узел } v_i \text{ инцидентен дуге } e_j \text{ и является её концом}, \\
-1, & \text{если узел } v_i \text{ инцидентен дуге } e_j \text{ и является её началом}, \\
0, & \text{если узел } v_i \text{ и дуга } e_j \text{ не инцидентны}.
\end{cases}
\]



Для матрицы инцидентций объём памяти:


\[
\eta(p, q) = \mathcal{O}(pq).
\]



\paragraph{Пример.} Матрицы инцидентций графов $G$ и $D$:



\[
G =
\begin{pmatrix}
1 & 0 & 0 & 0 & 1 \\
1 & 1 & 0 & 0 & 0 \\
0 & 1 & 1 & 0 & 0 \\
0 & 0 & 1 & 1 & 1
\end{pmatrix}
\qquad
D =
\begin{pmatrix}
-1 & 0 & 0 & 1 & 0 \\
1 & -1 & 0 & 0 & -1 \\
0 & 1 & 1 & 0 & 0 \\
0 & 0 & -1 & -1 & 1
\end{pmatrix}
\]



\paragraph{Замечание.}
Для связных графов $q > p$, поэтому матрица смежности несколько компактнее матрицы инцидентций.

\subsection*{7.4.4. Списки смежности}

Представление графа с помощью списочной структуры, отражающей смежность вершин и состоящей из массива указателей


\[
G : \text{array } [1..p] \text{ of } \uparrow N,
\]


на списки смежных вершин, где элемент списка описывается структурой:


\[
N = \text{record } \{ v : 1..p;\quad n : \uparrow N \} \text{ end record},
\]


называется \textbf{списком смежности}.

В случае представления неориентированных графов:


\[
\eta(p, q) = \mathcal{O}(p + 2q),
\]


а в случае ориентированных графов:


\[
\eta(p, q) = \mathcal{O}(p + q).
\]



\paragraph{Замечание.}
Массив $G$ также можно представить списком.

\paragraph{Пример.} Списки смежности для графа $G$ (слева) и орграфа $D$ (справа) представлены на рисунке ниже.

\begin{figure}[h]
\centering
\includegraphics[width=0.8\textwidth]{Photo/adjacency_lists_example.jpg}
\caption{Списки смежности для графа $G$ и орграфа $D$}
\end{figure}


\subsection*{7.4.5. Массив дуг}

Представление графа с помощью массива структур


\[
E : \text{array } [1..q] \text{ of record } \{ b, e : 1..p \} \text{ end record},
\]


отражающего список пар смежных вершин (или, для орграфов, узлов), называется \textbf{массивом рёбер} (или \textbf{массивом дуг}).

Для массива рёбер (или дуг) объём памяти:


\[
\eta(p, q) = \mathcal{O}(2q).
\]



\paragraph{Замечание.}
Для представления графов с изолированными вершинами может понадобиться хранить также число $p$, если только система программирования не позволяет извлечь это число из массива структур $E$.

\paragraph{Пример.}
Представление с помощью массива рёбер (дуг) показано в следующей таблице: слева — для графа $G$, справа — для орграфа $D$.

\begin{center}
\begin{tabular}{|c|c||c|c|}
\hline
\multicolumn{2}{|c||}{Граф $G$} & \multicolumn{2}{c|}{Орграф $D$} \\
\hline
$b$ & $e$ & $b$ & $e$ \\
\hline
1 & 2 & 1 & 2 \\
1 & 4 & 2 & 3 \\
2 & 3 & 2 & 4 \\
2 & 4 & 4 & 1 \\
3 & 4 & 4 & 3 \\
\hline
\end{tabular}
\end{center}

\paragraph{Замечание.}
Указанные представления пригодны для графов и орграфов, а после некоторой модификации — также и для псевдографов, мультиграфов и гиперграфов.

\subsection*{Обходы графов}

\textbf{Обход графа} — систематическое перечисление его вершин.

\subsection*{Теорема.}
Если граф $G$ связан и конечен, то поиск в ширину и поиск в глубину обходят все вершины по одному разу.

\paragraph{Суть методов:}
\begin{itemize}
  \item \textbf{Стек} — используется в поиске в глубину (DFS).
  \item \textbf{Очередь} — используется в поиске в ширину (BFS).
\end{itemize}

\paragraph{Алгоритм обхода графа:}
\begin{enumerate}
  \item Дан граф. Выбираем начальную вершину случайно или задаём вручную.
  \item Создаём массив, где отмечаем пройденные вершины (изначально все равны 0).
  \item Помещаем начальную вершину в структуру (стек или очередь) и отмечаем её.
  \item Заходим в цикл:
  \begin{enumerate}
    \item Извлекаем вершину из структуры.
    \item Просматриваем все смежные вершины:
    \begin{itemize}
      \item Если вершина не отмечена — помещаем её в структуру и отмечаем.
    \end{itemize}
    \item Повторяем, пока структура не опустеет.
  \end{enumerate}
\end{enumerate}


    \newpage
}
