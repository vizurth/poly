{
	\section{Гамильтоновы циклы. Теорема Дирака. Задача коммивояжера.}

	\section*{Гамильтоновы циклы. Теорема Дирака}

Граф называется \textbf{гамильтоновым}, если существует простой цикл, проходящий через все вершины графа ровно один раз.  
Такой цикл называется \textbf{гамильтоновым циклом}.

\subsection*{Теорема Дирака}

Пусть граф $G$ имеет $p$ вершин и минимальную степень $\delta(G)$.  
Если выполнено условие:


\[
\delta(G) \ge \frac{p}{2}
\]


то граф $G$ является гамильтоновым.

\textit{Примечание:} Простых необходимых и достаточных условий гамильтоновости не существует, однако теорема Дирака даёт полезный критерий.

\section*{Задача коммивояжера}

Имеется $p$ городов, расстояния между которыми известны.  
Коммивояжер должен посетить каждый город ровно один раз и вернуться в исходный.  
Требуется найти маршрут, минимизирующий суммарное расстояние.

Это задача нахождения \textbf{кратчайшего гамильтонова цикла} в нагруженном полном графе.

\textbf{Сложность:}  
Задача является NP-трудной.  
Точный алгоритм возможен только методом полного перебора.  
На практике применяются эвристические и аппроксимационные методы.


    \newpage
}
