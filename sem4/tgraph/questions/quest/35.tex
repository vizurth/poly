{
	\section{Элементы сетевого планирования: критические пути, работы, резервы. Линейные
графики. Алгоритм сетевого планирования (составлял гпт)}

\section*{Элементы сетевого планирования}

\subsection*{Критические пути}

\textbf{Критический путь} — это последовательность работ, определяющая минимальную длительность выполнения всего проекта.  
Любое изменение длительности работ на критическом пути напрямую влияет на срок завершения проекта.

\begin{itemize}
  \item Критический путь — путь с нулевым полным резервом.
  \item Может быть не единственным.
  \item Определяется по графу проекта.
\end{itemize}

\subsection*{Работы}

\textbf{Работа} — элемент проекта, имеющий продолжительность и связывающий две события (вехи).  
В сетевом графе представляется ориентированным ребром.

\begin{itemize}
  \item Каждая работа имеет раннее и позднее время начала и окончания.
  \item Работы могут быть реальными (физическими) и фиктивными (логическими связями).
\end{itemize}

\subsection*{Резервы времени}

\begin{itemize}
  \item \textbf{Полный резерв} — максимальное время, на которое можно задержать выполнение работы без изменения срока завершения проекта.
  \item \textbf{Свободный резерв} — время, на которое можно задержать работу без влияния на раннее начало последующих работ.
  \item \textbf{Нулевой резерв} — признак принадлежности работы к критическому пути.
\end{itemize}

\subsection*{Линейные графики}

\textbf{Линейный график} — способ визуализации выполнения работ во времени.  
Оси: горизонтальная — время, вертикальная — работы.

\begin{itemize}
  \item Позволяет оценить загрузку ресурсов.
  \item Удобен для строительных и производственных процессов.
\end{itemize}

\subsection*{Алгоритм сетевого планирования}

\begin{enumerate}
  \item Построение сетевого графа: определение событий и работ.
  \item Расчёт ранних сроков начала и окончания работ.
  \item Расчёт поздних сроков начала и окончания работ.
  \item Вычисление резервов времени.
  \item Определение критического пути.
  \item Построение календарного плана и линейного графика.
\end{enumerate}


    \newpage
}
