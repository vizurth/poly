{
	\section{Деревья. Свободные деревья. Основные свойства деревьев (с доказательствами).
Код Прюфера.}

\section*{Деревья и их свойства}

\subsection*{Определения}

\begin{itemize}
  \item \textbf{Свободное дерево} — связный ациклический граф.
  \item \textbf{Древочисленный граф} — граф, для которого выполняется равенство:
  

\[
  q(G) = p(G) - 1,
  \]


  где $p(G)$ — число вершин, $q(G)$ — число рёбер.
\end{itemize}

\subsection*{Теорема}

Пусть $G(V, E)$ — граф с $p$ вершинами, $q$ рёбрами, $k$ компонентами связности и $z$ простыми циклами.  
Тогда любые два из следующих четырёх свойств:
\begin{enumerate}
  \item связность;
  \item ацикличность;
  \item древочисленность ($q = p - 1$);
  \item субцикличность ($z = 0$),
\end{enumerate}
в совокупности характеризуют граф как дерево.

\section*{Характеризации графов}

\begin{enumerate}
  \item $G$ — дерево, то есть связный граф без циклов:  
  

\[
  k(G) = 1,\quad z(G) = 0
  \]



  \item Любые две вершины соединены в $G$ единственной простой цепью:  
  

\[
  \forall u, v \in V,\quad |P(u, v)| = 1
  \]



  \item $G$ — связный граф, и каждое ребро является мостом:  
  

\[
  k(G) = 1,\quad \forall e \in E,\quad k(G - e) > 1
  \]



  \item $G$ — связный и древочисленный граф:  
  

\[
  k(G) = 1,\quad q(G) = p(G) - 1
  \]



  \item $G$ — ациклический и древочисленный граф:  
  

\[
  z(G) = 0,\quad q(G) = p(G) - 1
  \]



  \item $G$ — ациклический и субциклический граф:  
  

\[
  z(G) = 0,\quad z(G + x) = 1
  \]



  \item $G$ — связный, субциклический и неполный граф:  
  

\[
  k(G) = 1,\quad G \ne K_p,\quad p \ge 3,\quad z(G + x) = 1
  \]



  \item $G$ — древочисленный и субциклический граф (за двумя исключениями):  
  

\[
  q(G) = p(G) - 1 \land G \ne K_1 \cup K_3 \land G \ne K_2 \cup K_3 \land z(G + x) = 1
  \]


\end{enumerate}

\section*{Следствия из свойств графов}

\textbf{Следствие 1.}  
В любом нетривиальном дереве имеются по крайней мере две висячие вершины.  
Например, это могут быть концы диаметра дерева.

\vspace{1em}

\textbf{Следствие 2.}  
Каждая невисячая вершина свободного дерева является точкой сочленения.

\vspace{1em}

\textbf{Следствие 3.}  
Если в связном графе нет висячих вершин, то в нём обязательно существует цикл.

\section*{Код Прюфера}

\begin{figure}[h]
\centering
\includegraphics[width=0.8\textwidth]{Photo/ALGO_18_1.jpg}
% \caption{Списки смежности для графа $G$ и орграфа $D$}
\end{figure}


\begin{figure}[h]
\centering
\includegraphics[width=0.8\textwidth]{Photo/ALGO_18_2.jpg}
% \caption{Списки смежности для графа $G$ и орграфа $D$}
\end{figure}


    \newpage
}
