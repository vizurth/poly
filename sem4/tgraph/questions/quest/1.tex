{
	\section{Специальные бинарные отношения. Связь между понятием отношения и понятием
графа. Орграфы и бинарные отношения.}


\subsection*{Бинарные отношения}
Пусть $A$ — непустое множество. \textbf{Бинарным отношением} на $A$ называется любое подмножество


\[
R \subseteq A \times B.
\]


Пару $(a,b) \in R$ принято обозначать как $a\,R\,b$.

\subsection*{Специальные свойства бинарных отношений}

\begin{itemize}
    \item \textbf{Рефлексивность:} 
    

\[
    \forall a \in A:\ (a,a) \in R.
    \]



    \item \textbf{Антирефлексивность:}
    

\[
    \forall a \in A:\ (a,a) \notin R.
    \]



    \item \textbf{Симметричность:}
    

\[
    \forall a,b \in A:\ (a,b) \in R \Rightarrow (b,a) \in R.
    \]



    \item \textbf{Антисимметричность:}
    

\[
    (a,b)\in R \ \text{и}\ (b,a)\in R \Rightarrow a=b.
    \]



    \item \textbf{Транзитивность:}
    

\[
    (a,b)\in R \ \text{и}\ (b,c)\in R \Rightarrow (a,c)\in R.
    \]


\end{itemize}

\subsection*{Основное определение}

Графом $G(V, E)$ называется совокупность двух множеств — непустого множества $V$ (множества вершин) и множества $E$ двухэлементных подмножеств множества $V$ ($E$ — множество рёбер),


\[
G(V, E) \overset{\text{def}}{=} \langle V; E \rangle,\quad V \neq \emptyset,\quad E \subseteq 2^V\ \&\ \forall e \in E\ (|e| = 2).
\]



\paragraph{ЗАМЕЧАНИЕ.}
Легко видеть, что любое множество $E$ двухэлементных подмножеств множества $V$ определяет симметричное бинарное отношение на множестве $V$. Поэтому можно считать, что


\[
E \subseteq V \times V,\quad E = E^{-1}
\]


и трактовать ребро не только как множество $\{v_1, v_2\}$, но и как пару $(v_1, v_2)$.

\smallskip

Число вершин графа $G$ обозначим $p$, а число рёбер — $q$:


\[
p \overset{\text{def}}{=} p(G) \overset{\text{def}}{=} |V|,\quad q \overset{\text{def}}{=} q(G) \overset{\text{def}}{=} |E|.
\]



Если хотят явно упомянуть числовые характеристики графа, то говорят: $(p, q)$-граф.


\subsection*{Ориентированный граф (орграф)}

Если элементами множества $E$ являются упорядоченные пары (т. е. $E \subseteq V \times V$), то граф называется \textbf{ориентированным} (или \textbf{орграфом}). В этом случае элементы множества $V$ называются \textbf{узлами}, а элементы множества $E$ — \textbf{дугами}.


\subsection*{Орграфы и свойства отношений}

Свойства отношения естественно интерпретируются в терминах орграфов:

\begin{itemize}
    \item \textbf{Рефлексивность} соответствует наличию петли в каждой вершине.
    \item \textbf{Антирефлексивность} означает отсутствие петель.
    \item \textbf{Симметричность} означает, что каждая дуга имеет дугу в обратном направлении.
    \item \textbf{Антисимметричность} означает отсутствие пар противоположных дуг между различными вершинами.
    \item \textbf{Транзитивность} означает: если есть дуги $a \to b$ и $b \to c$, то должна быть дуга $a \to c$.
\end{itemize}

\subsection*{Соответствие графов и отношений}

\begin{itemize}
    \item Полный граф — универсальное отношение.
    \item Неорграф — симметричное отношение.
    \item Дополнение графов — дополнение отношений.
    \item Изменение всех направлений дуг — обратное отношение.
\end{itemize}

\subsection*{Итог}
Бинарное отношение на множестве $A$ полностью эквивалентно ориентированному графу на том же множестве вершин. Свойства отношения имеют естественные графовые интерпретации, что делает орграфы удобным инструментом для визуализации и анализа отношений.


    \newpage
}
