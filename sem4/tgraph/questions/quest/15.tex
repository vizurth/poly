{
	\section{Поток минимальной стоимости. Алгоритм определения потока минимальной
стоимости}

\subsection*{Поток минимальной стоимости}

Пусть задана транспортная сеть $G(V, E)$, где каждому ребру $e \in E$ сопоставлены:
\begin{itemize}
  \item пропускная способность $c(e)$;
  \item стоимость единицы потока $w(e)$.
\end{itemize}

Требуется найти поток максимальной величины или фиксированного объёма $F$,  
\textbf{минимизирующий суммарную стоимость}:



\[
\text{Cost}(f) = \sum_{e \in E} f(e) \cdot w(e)
\]



при выполнении условий:
\begin{itemize}
  \item ограничения пропускной способности: \quad $0 \le f(e) \le c(e)$;
  \item закон сохранения потока во всех вершинах, кроме истока и стока.
\end{itemize}

Такой поток называется \textbf{потоком минимальной стоимости}.

\subsection*{Основная идея}

Если рассматривать стоимость как вес рёбер, то задача сводится к поиску путей минимальной стоимости в остаточной сети.  
Пока существует путь, уменьшающий стоимость потока, поток можно улучшать.

\subsection*{Алгоритм нахождения потока минимальной стоимости}

Один из классических методов — \textbf{алгоритм наименьшей стоимости увеличивающего пути} (Successive Shortest Path).

\begin{enumerate}
  \item Инициализировать поток $f = 0$.
  \item Построить остаточную сеть $G_f$:
  \begin{itemize}
    \item прямые рёбра имеют стоимость $w(e)$;
    \item обратные рёбра имеют стоимость $-w(e)$.
  \end{itemize}
  \item Найти кратчайший путь по стоимости из истока $s$ в сток $t$ в остаточной сети.
  \item Если пути нет — текущий поток является потоком минимальной стоимости.
  \item Иначе:
  \begin{itemize}
    \item определить возможное увеличение $\Delta$ по минимальной пропускной способности на пути;
    \item увеличить поток вдоль пути на $\Delta$;
    \item обновить остаточную сеть.
  \end{itemize}
  \item Перейти к шагу 3.
\end{enumerate}

\subsection*{Свойства}

\begin{itemize}
  \item Алгоритм всегда завершается, если все стоимости целые.
  \item При использовании алгоритма Дейкстры с потенциалами (метод Джонсона) достигается полиномиальная сложность.
  \item Метод применяется в логистике, распределении ресурсов, оптимизации расписаний.
\end{itemize}

\subsection*{Альтернативные методы}

\begin{itemize}
  \item Метод потенциалов (алгоритм Беллмана–Форда + корректировка весов).
  \item Алгоритм цикла отрицательной стоимости (canceling negative cycles).
  \item Линейное программирование (решение через симплекс-метод).
\end{itemize}


    \newpage
}
