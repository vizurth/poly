{
	\section{Выровненные, заполненные и полные деревья. Сбалансированные деревья.
Алгоритм бинарного (двоичного) поиска.}

\section*{Типы бинарных деревьев}

\begin{itemize}
  \item \textbf{Выровненное бинарное дерево} — все листья находятся на одном (последнем) уровне.

  \item \textbf{Заполненное бинарное дерево} — все узлы, степень которых меньше максимальной, располагаются на одном или двух последних ярусах. То есть все ярусы, кроме последнего, полностью заполнены.

  \item \textbf{Полное бинарное дерево} — все ярусы заполнены, все узлы имеют полустепень исхода $2$, листья находятся на последнем уровне.  
  Такое дерево содержит $2^{h - 1}$ узлов, где $h$ — высота дерева.

  \item \textbf{Сбалансированное дерево} — для любого узла высота левого и правого поддеревьев отличается не более чем на единицу.  
  Возможна также балансировка по весу.  
  Сбалансированные деревья менее эффективны по времени поиска, чем полные, но проще в реализации операций вставки и удаления.
\end{itemize}

\section*{Алгоритм бинарного поиска}

В отсортированном массиве:
\begin{enumerate}
  \item Выбирается середина.
  \item Сравнивается искомое значение с серединным.
  \item В зависимости от результата поиск продолжается в левой или правой половине массива.
\end{enumerate}

\begin{figure}[h]
\centering
\includegraphics[width=0.8\textwidth]{Photo/algo_22.jpg}
% \caption{Списки смежности для графа $G$ и орграфа $D$}
\end{figure}



    \newpage
}
