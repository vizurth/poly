{
	\section{Элементы графов: подграфы, валентность, маршруты, цепи, циклы. Метрические
характеристики графа. Особенности алгоритмов теории графов}

\subsection*{Подграфы}

Граф $G'(V', E')$ называется \textbf{подграфом} (или \textbf{частью}) графа $G(V, E)$ (обозначается $G' \subseteq G$), если


\[
V' \subseteq V \quad \& \quad E' \subseteq E.
\]



Если $V' = V$, то $G'$ называется \textbf{остовным подграфом} графа $G$.

Если $V' \subset V$, $E' \subset E$ и $(V' \neq V \lor E' \neq E)$, то граф $G'$ называется \textbf{собственным подграфом} графа $G$.

Подграф $G'(V', E')$ называется \textbf{правильным подграфом} графа $G(V, E)$, если он содержит все возможные рёбра графа $G$ между вершинами из $V'$:


\[
\forall \, u, v \in V'\quad ((u, v) \in E \Rightarrow (u, v) \in E').
\]



Правильный подграф $G'(V', E')$ графа $G(V, E)$ определяется подмножеством вершин $V'$.

\subsection*{Замечание.}
Иногда подграфами называют только правильные подграфы, а неправильные подграфы называют \textbf{изграфами}.


\subsection*{Степень вершины}

Количество рёбер, инцидентных вершине $v$, называется \textbf{степенью} (или \textbf{валентностью}) вершины $v$ и обозначается $d(v)$:


\[
\forall v \in V\quad 0 \leq d(v) \leq p - 1,\qquad d(v) = |\Gamma^+(v)|.
\]



Таким образом, степень $d(v)$ вершины $v$ совпадает с количеством смежных с ней вершин. Количество вершин, не смежных с $v$, обозначается $\overline{d}(v)$. Ясно, что:


\[
\forall v \in V\quad d(v) + \overline{d}(v) = p - 1.
\]



Обозначим минимальную степень вершины графа $G$ через $\delta(G)$, а максимальную — через $\Delta(G)$:


\[
\delta(G(V, E)) \overset{\text{def}}{=} \min_{v \in V} d(v),\qquad
\Delta(G(V, E)) \overset{\text{def}}{=} \max_{v \in V} d(v).
\]



Очевидно, что $\delta(G)$ и $\Delta(G)$ являются \textbf{инвариантами} графа.

Если степени всех вершин равны $k$, то граф называется \textbf{регулярным степени $k$}:


\[
\delta(G) = \Delta(G) = k,\qquad \forall v \in V\quad d(v) = k.
\]



Степень регулярности обозначается $r(G)$. Для нерегулярных графов $r(G)$ не определено.

\subsection*{Маршруты, цепи, циклы}

\textbf{Маршрутом} в графе называется чередующаяся последовательность вершин и рёбер, начинающаяся и заканчивающаяся вершиной:


\[
v_0, e_1, v_1, e_2, v_2, \dots, e_k, v_k,
\]


в которой любые два соседних элемента инцидентны, причём однородные элементы (вершины, рёбра) через один — смежны или совпадают.

\subsection*{Маршруты, цепи, циклы}

Если $v_0 = v_k$, то маршрут называется \textbf{замкнутым}, иначе — \textbf{открытым}.

Если все рёбра различны, то маршрут называется \textbf{цепью}.  
Если все вершины (а значит, и рёбра) различны, то маршрут называется \textbf{простой цепью}.

В цепи $v_0, e_1, \ldots, e_k, v_k$ вершины $v_0$ и $v_k$ называются \textbf{концами цепи}.  
Говорят, что цепь с концами $u$ и $v$ \textbf{соединяет вершины} $u$ и $v$.

Цепь, соединяющая вершины $u$ и $v$, обозначается $\langle u, v \rangle$.  
Если нужно указать граф $G$, которому принадлежит цепь, то добавляют индекс: $\langle u, v \rangle_G$.

Нетрудно показать, что если существует какая-либо цепь, соединяющая вершины $u$ и $v$, то существует и \textbf{простая цепь}, соединяющая эти вершины.

\textbf{Замкнутая цепь} называется \textbf{циклом};  
\textbf{замкнутая простая цепь} называется \textbf{простым циклом}.

Число циклов в графе $G$ обозначается $z(G)$.  
Граф без циклов называется \textbf{ациклическим}. \\ 

Для орграфов \emph{цепь} называется \emph{путём}, а \emph{цикл} — \emph{контуром}.

Путь в орграфе из узла $u$ в узел $v$ обозначается:


\[
\langle \vec{u}, v \rangle.
\]

\subsection*{Метрические характеристики графа}

\textbf{Длина маршрута} — количество рёбер в нём.  
Маршрут $M$ имеет длину $k$ тогда и только тогда, когда $|M| = k$.

\textbf{Расстоянием} между вершинами $u$ и $v$, обозначаемым $d(u, v)$, называется длина кратчайшей цепи $\langle u, v \rangle$, а сама кратчайшая цепь называется \emph{геодезической}:


\[
d(u, v) = \min_{\{\langle u, v \rangle\}} |\langle u, v \rangle|.
\]


Если цепи нет, расстояние считается бесконечным.

\textbf{Ярус} — множество вершин на расстоянии $n$ от вершины $v$.

\textbf{Диаметр графа} — длина самой длинной геодезической цепи:


\[
D(G) = \max_{u, v \in V} d(u, v).
\]



\textbf{Эксцентриситет} $e(v)$ вершины $v$ в связном графе — максимальное расстояние от $v$ до других вершин.

\textbf{Радиус графа} $R(G)$ — наименьший эксцентриситет:


\[
R(G) = \min_{v \in V} e(v).
\]



Вершина называется \emph{центральной}, если её эксцентриситет совпадает с радиусом.  
Множество центральных вершин называется \emph{центром графа}.



    \newpage
}
