{
	\section{Разметка графа. Грациозная, счастливая разметки. Раскраска графа. Примеры
задач. Хроматическое число. Алгоритмы раскрашивания. Двойственный граф}

\section*{Разметка графа. Грациозная, счастливая разметки}

\textbf{Разметка} — это функция из множества вершин/рёбер графа в множество меток.

\subsection*{Грациозная разметка}

Грациозная разметка вершин:  
Если все вершины графа помечены числами от $0$ до $q$, и эта разметка порождает рёберную разметку от $1$ до $q$.

Иначе говоря, для любого ребра между двумя вершинами модуль разности их меток должен быть равен метке ребра:


\[
\text{метка ребра} = |\text{метка вершины}_1 - \text{метка вершины}_2|
\]



\subsection*{Счастливая разметка}

Счастливая разметка:  
Назначаем положительные целые числа вершинам так, чтобы сумма меток соседних вершин соответствовала раскраске графа.

\section*{Раскраска графа}

Цель раскраски — назначить цвета вершинам графа так, чтобы никакие две смежные вершины не имели одинаковый цвет.

\begin{itemize}
  \item \textbf{Хроматическое число} графа $\chi(G)$ — минимальное количество цветов, необходимое для корректной раскраски вершин.
  \item \textbf{Хроматический индекс} графа $\chi'(G)$ — минимальное количество цветов, необходимое для корректной раскраски рёбер.
\end{itemize}

\subsection*{Примеры задач}

\begin{enumerate}
  \item Составление расписаний: лекции, которые нельзя проводить одновременно, соединяются рёбрами.
  \item Распределение оборудования: аналогично, конфликты моделируются рёбрами.
  \item Раскраска карты: соседние страны должны иметь разные цвета.
\end{enumerate}

\section*{Алгоритмы раскрашивания}

\begin{itemize}
  \item Выбираем максимальное независимое множество — раскрашиваем его, затем удаляем эти вершины и повторяем.
  \item Задача раскраски является NP-полной.
  \item Возможна эвристика: начинать с вершин с наибольшей степенью.
\end{itemize}

\section*{Двойственный граф}

В двойственном графе:
\begin{itemize}
  \item Вершины соответствуют граням исходного графа;
  \item Рёбра проводятся между вершинами, если соответствующие грани имеют общее ребро.
\end{itemize}



    \newpage
}
