{
	\section{Связность: компоненты связности, точки сочленения. Вершинная и реберная
связность (мосты и блоки, меры связности).}

\section*{Связность: компоненты связности, точки сочленения}

\subsection*{Теорема}
Граф связан тогда и только тогда, когда его нельзя представить в виде объединения двух графов.

\subsection*{Компоненты связности}
Классы эквивалентности по отношению связности называются \textbf{компонентами связности} графа.  
Число компонент связности обозначается $k(\mathcal{G})$.

\subsection*{Точка сочленения}
Вершина графа называется \textbf{точкой сочленения}, если её удаление увеличивает число компонент связности.

\subsection*{Мост}
\textbf{Мостом} называется ребро, удаление которого увеличивает число компонент связности.

\paragraph{Замечание}
В любом нетривиальном графе существует по крайней мере две вершины, которые не являются точками сочленения.

\section*{Вершинная и реберная связность (мосты и блоки, меры связности)}

\subsection*{Теорема 1}
Пусть $G(V, E)$ — связный граф и $v \in V$. Тогда следующие утверждения эквивалентны:
\begin{enumerate}
  \item $v$ — точка сочленения.
  \item $\exists\, u, w \in V$ такие, что $u \ne w$ и $v \in \langle u, w \rangle_G$.
  \item $\exists\, U, W \subseteq V \setminus \{v\}$ такие, что $U \cap W = \emptyset$, $U \cup W = V \setminus \{v\}$ и для всех $u \in U$, $w \in W$ любые пути $\langle u, w \rangle_G$ проходят через $v$.
\end{enumerate}

\subsection*{Следствие}
Если вершина инцидентна мосту и не является висячей, то она является точкой сочленения.

\subsection*{Теорема 2}
Пусть $G(V, E)$ — связный граф и $x \in E$. Тогда следующие утверждения эквивалентны:
\begin{enumerate}
  \item $x$ — мост.
  \item $x$ не принадлежит ни одному простому циклу.
  \item $\exists\, u, w \in V$ такие, что все пути $\langle u, w \rangle_G$ содержат $x$.
  \item $\exists\, U, W \subseteq V$ такие, что $U \cap W = \emptyset$, $U \cup W = V$ и для всех $u \in U$, $w \in W$ любые пути $\langle u, w \rangle_G$ содержат $x$.
\end{enumerate}

\subsection*{Вершинная связность}
\textbf{Вершинной связностью} графа называется наименьшее число вершин, удаление которых приводит к несвязному или тривиальному графу. Обозначается $\chi(G)$.

\subsection*{Рёберная связность}
\textbf{Рёберной связностью} графа называется наименьшее число рёбер, удаление которых приводит к несвязному или тривиальному графу. Обозначается $\lambda(G)$.




    \newpage
}
