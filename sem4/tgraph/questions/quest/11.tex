{
	\section{Нахождение кратчайших путей: алгоритм Дейкстры, алгоритм Беллмана-Форда.}

	\section*{Алгоритм Дейкстры}

Алгоритм Дейкстры используется для поиска кратчайших путей от заданной вершины-истока $s$ до всех остальных вершин графа.

\subsection*{Сложность}


\[
\mathcal{O}(p^2)
\]



\subsection*{Результат}
Алгоритм возвращает вектор расстояний $d[v]$ от истока $s$ до каждой вершины $v \in V$.

\subsection*{Шаги алгоритма}

\begin{enumerate}
  \item \textbf{Инициализация:}
  \begin{itemize}
    \item Для всех вершин $v \in V$ задать $d[v] := \infty$.
    \item Для истока $s$ задать $d[s] := 0$.
    \item Создать множество необработанных вершин $Q := V$.
  \end{itemize}

  \item \textbf{Основной цикл:} Пока $Q \ne \emptyset$:
  \begin{enumerate}
    \item Выбрать вершину $u \in Q$ с минимальным значением $d[u]$.
    \item Удалить $u$ из $Q$.
    \item Для каждого соседа $v$ вершины $u$, где $v \in Q$:
    \begin{itemize}
      \item Вычислить новое расстояние: $\text{temp} := d[u] + w(u, v)$.
      \item Если $\text{temp} < d[v]$, то обновить: $d[v] := \text{temp}$.
    \end{itemize}
  \end{enumerate}
\end{enumerate}

\section*{Алгоритм Беллмана–Форда}

Алгоритм Беллмана–Форда используется для поиска кратчайших путей от заданной вершины-истока $s$ до всех остальных вершин графа, допускающего отрицательные веса рёбер.

\subsection*{Сложность}


\[
\mathcal{O}(pq)
\]



\subsection*{Результат}
Алгоритм возвращает вектор расстояний $d[v]$ от истока $s$ до каждой вершины $v \in V$.

\subsection*{Инициализация}
\begin{itemize}
  \item Для всех вершин $v \in V$ задать $d[v] := \infty$.
  \item Для истока $s$ задать $d[s] := 0$.
\end{itemize}

\subsection*{Основной алгоритм}
\begin{enumerate}
  \item Повторить следующие шаги $|V| - 1$ раз:
  \begin{itemize}
    \item Для каждого ребра $(u, v) \in E$:
    \begin{itemize}
      \item Вычислить новое расстояние: $\text{temp} := d[u] + w(u, v)$.
      \item Если $\text{temp} < d[v]$, то обновить: $d[v] := \text{temp}$.
    \end{itemize}
  \end{itemize}
\end{enumerate}

\subsection*{Проверка на наличие отрицательных циклов}
\begin{itemize}
  \item Для каждого ребра $(u, v) \in E$:
  \begin{itemize}
    \item Если $d[u] + w(u, v) < d[v]$, то в графе существует отрицательный цикл.
  \end{itemize}
\end{itemize}



    \newpage
}
