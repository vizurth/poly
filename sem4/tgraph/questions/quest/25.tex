{
	\section{Фундаментальные циклы и разрезы. Фундаментальная система циклов и
циклический ранг. Фундаментальная система разрезов и коциклический ранг.
Подпространства циклов и коциклов}
Рассматриваются только \textbf{связные графы}.


\section*{Циклы и разрезы}

\subsection*{Циклы}

Цикл может входить только в одну компоненту связности графа $G(V, E)$.  
В несвязном графе понятие разреза является вырожденным, поэтому без ограничения общности в этом разделе граф $G(V, E)$ считается связным.

Цикл не может содержать одно и то же ребро более одного раза, поэтому в данном контексте цикл рассматривается как \emph{множество рёбер}.  
В связи с этим можно дать эквивалентное определение:

\textbf{Простым циклом} называется такой цикл, никакое собственное подмножество которого не является циклом.


\subsection*{Разрез}

\textbf{Разрезом} называется множество рёбер, удаление которых делает граф несвязным.  
Любое разбиение множества вершин $V$ на два непустых подмножества $V_1$ и $V_2$ определяет разрез:


\[
S := \{(v_1, v_2) \in E \mid v_1 \in V_1,\ v_2 \in V_2\}
\]



Очевидно, что множества $V_1$ и $V_2$ определяют друг друга:


\[
V_1 = V \setminus V_2,\quad V_2 = V \setminus V_1
\]


Поэтому достаточно задать одно из них:


\[
\forall U \subset V,\quad \overline{U} \overset{\text{def}}{=} V \setminus U
\]



\subsection*{Обозначение}

Обозначим через $E(V_1, V_2)$ множество рёбер, соединяющих два дизъюнктивных непустых подмножества вершин графа $G = (V, E)$:


\[
E(V_1, V_2) \overset{\text{def}}{=} \{(v_1, v_2) \in E \mid v_1 \in V_1,\ v_2 \in V_2\}
\]



\subsection*{Правильный разрез}

Разрез связного графа $G(V, E)$, определяемый непустым подмножеством $U \subset V$, называется \textbf{правильным разрезом} и обозначается:


\[
S(U) \overset{\text{def}}{=} E(U, \overline{U})
\]



Правильный разрез не содержит лишних рёбер и соответствует минимальному разделению графа по вершинам.

\section*{Система циклов и ранг}

Пусть $T(V, E_T)$ — остов графа $G(V, E)$.  
\textbf{Кодеревом} $T^*(V, E_T^*)$ остова $T$ называется остовный подграф, такой что:


\[
E_T^* = E \setminus E_T
\]


Кодерево не является деревом. Рёбра кодерева называются \emph{хордами} остова.

Каждая хорда $e \in E_T^*$ порождает ровно один простой цикл, обозначаемый $Z_e$.  
Таким образом, множество таких циклов образует \textbf{фундаментальную систему циклов}:


\[
\mathcal{Z} \overset{\text{def}}{=} \{Z_e\}_{e \in T^*}
\]



Количество циклов в фундаментальной системе называется \textbf{циклическим рангом} графа и обозначается $m(G)$.

\subsection*{Теорема}

Любой цикл в связном графе можно представить как \emph{симметрическую разность} нескольких фундаментальных циклов, определяемых произвольным остовом $T$.

Циклический ранг графа равен числу хорд остова:


\[
m(G) = q - p + 1
\]


где $q = |E|$ — число рёбер, $p = |V|$ — число вершин графа $G$.

\section*{Фундаментальная система разрезов и коциклический ранг}

Пусть $T(V, E_T)$ — остов графа $G(V, E)$.  
Рассмотрим ребро $e \in E_T$ и определим разрез $S_e$ следующим образом.

Ребро $e$ является мостом в дереве $T$, поэтому его удаление разбивает множество вершин $V$ на два непустых подмножества $V_1$ и $V_2$.  
Включим в разрез $S_e$ само ребро $e$ и все хорды остова $T$, соединяющие вершины из $V_1$ с вершинами из $V_2$:


\[
S_e \overset{\text{def}}{=} \{e\} \cup \{(v_1, v_2) \in T^* \mid v_1 \in V_1,\ v_2 \in V_2\} = E(V_1, V_2)
\]



Такой разрез $S_e$ называется \textbf{простым разрезом}.

Множество всех таких разрезов, порождённых рёбрами остова, образует \textbf{фундаментальную систему разрезов}:


\[
\mathcal{S} \overset{\text{def}}{=} \{S_e\}_{e \in T}
\]



Количество разрезов в фундаментальной системе называется \textbf{коциклическим рангом} графа и обозначается $m^*(G)$.

\subsection*{Теорема}

Любой \emph{правильный разрез} в связном графе можно представить как \emph{симметрическую разность} нескольких фундаментальных разрезов, определяемых произвольным остовом $T$.

\subsection*{Формула ранга}

Число разрезов в фундаментальной системе равно числу рёбер остова:


\[
m^*(G) = p - 1
\]


где $p = |V|$ — число вершин графа $G$.

\section*{Подпространства циклов и коциклов}

Рассмотрим граф $G(V, E)$, где множество рёбер $E$ интерпретируется как векторное пространство над бинарной арифметикой (по модулю 2).  
Элементы этого пространства можно отождествить с подмножествами рёбер: ребро входит в подмножество тогда и только тогда, когда его коэффициент в линейной комбинации равен $1$.

\subsection*{Операции}

Сложение векторов определяется как \textbf{симметрическая разность} множеств рёбер:


\[
A \oplus B := (A \cup B) \setminus (A \cap B)
\]



\subsection*{Циклическое и коциклическое подпространства}

Множества циклов и правильных разрезов замкнуты относительно симметрической разности, поэтому они образуют подпространства в пространстве всех подмножеств рёбер графа.

\begin{itemize}
  \item Циклы рассматриваются как \emph{циклические векторы}.
  \item Правильные разрезы — как \emph{коциклические векторы}.
\end{itemize}

\subsection*{Базисы и ранги}

\begin{itemize}
  \item \textbf{Независимая система циклов} — такая система $\{Z_i\}$, в которой ни один цикл не выражается как линейная комбинация остальных.
  \item Максимальная независимая система циклов образует \textbf{базис} циклического подпространства.
  \item \textbf{Фундаментальная система циклов}, порождённая остовом, является таким базисом.
  \item \textbf{Циклический ранг} графа $m(G)$ — это размерность циклического подпространства.

  \item Аналогично, максимальная независимая система правильных разрезов образует \textbf{базис} коциклического подпространства.
  \item \textbf{Фундаментальная система разрезов}, порождённая остовом, является таким базисом.
  \item \textbf{Коциклический ранг} графа $m^*(G)$ — это размерность коциклического подпространства.
\end{itemize}



    \newpage
}
