{
	\section{Независимые и покрывающие множества вершин и ребер. Теорема о связи чисел
независимости и покрытий}

\section*{Независимые и покрывающие множества вершин и рёбер}

\begin{itemize}
  \item Множество вершин называется \textbf{независимым}, если никакие две вершины из него не смежны.  
  Максимальное число вершин в независимом множестве называется \textbf{числом независимости по вершинам} и обозначается $\beta_0$.

  \item Множество рёбер называется \textbf{независимым} (или \textbf{паросочетанием}), если никакие два ребра не смежны.  
  Максимальное число рёбер в независимом множестве называется \textbf{числом независимости по рёбрам} и обозначается $\beta_1$.

  \item Вершина \textbf{покрывает} свои инцидентные рёбра, а ребро — свои инцидентные вершины.

  \item Множество вершин, покрывающее все рёбра графа, называется \textbf{покрывающим множеством вершин}.  
  Минимальное число таких вершин называется \textbf{числом покрытия по вершинам} и обозначается $\alpha_0$.

  \item Множество рёбер, покрывающее все вершины графа, называется \textbf{покрывающим множеством рёбер}.  
  Минимальное число таких рёбер называется \textbf{числом покрытия по рёбрам} и обозначается $\alpha_1$.
\end{itemize}

\section*{Теорема о связи чисел независимости и покрытий}

Для любого нетривиального графа выполняется равенство:


\[
\alpha_0 + \beta_0 = \alpha_1 + \beta_1
\]




    \newpage
}
