{
	\section{Нахождение кратчайших путей: алгоритм Флойда-Уоршалла. Алгоритм
нахождения максимального пути.}

\section*{Алгоритм Флойда–Уоршалла}

Алгоритм Флойда–Уоршалла используется для нахождения кратчайших расстояний между всеми парами вершин во взвешенном графе (возможно с отрицательными весами, но без отрицательных циклов).

\subsection*{Сложность}


\[
\mathcal{O}(p^3)
\]



\subsection*{Результат}
Алгоритм возвращает матрицу расстояний размером $p \times p$.

\subsection*{Инициализация}
\begin{itemize}
  \item Для всех пар вершин $(i, j)$ задать $d[i][j] := \infty$.
  \item Для всех $i$ задать $d[i][i] := 0$.
  \item Для каждого ребра $(u, v)$ с весом $w$ задать $d[u][v] := w$.
\end{itemize}

\subsection*{Основной алгоритм}
\begin{itemize}
  \item Для каждой вершины $k$ от $1$ до $n$:
  \begin{itemize}
    \item Для каждой пары вершин $(i, j)$:
    \begin{itemize}
      \item Вычислить: $\text{temp} := d[i][k] + d[k][j]$.
      \item Если $\text{temp} < d[i][j]$, то обновить: $d[i][j] := \text{temp}$.
    \end{itemize}
  \end{itemize}
\end{itemize}

\subsection*{Проверка на отрицательные циклы (опционально)}
Если после выполнения алгоритма существует $i$ такое, что $d[i][i] < 0$, то в графе присутствует отрицательный цикл.

\section*{Алгоритм нахождения максимального пути}

Для нахождения максимального пути в графе можно использовать полный перебор всех возможных путей от текущей вершины ко всем достижимым из неё вершинам.

\subsection*{Идея}
Перебрать все возможные пути от текущей вершины до всех последующих, достижимых из неё, и выбрать путь с максимальной суммарной длиной (весом).

\subsection*{Применение}
Подходит для ориентированных ациклических графов (DAG), где отсутствуют циклы, и можно использовать динамическое программирование или топологическую сортировку для оптимизации.



    \newpage
}
