{
	\section{Планарность. Укладка графов. Эйлерова характеристика. Теорема о пяти красках.}


\subsection*{Укладка графов}

Уложить граф на поверхности — значит изобразить его так, чтобы рёбра не пересекались (кроме в вершинах).  
Если это возможно на плоскости, граф называется \textbf{планарным}.

\begin{itemize}
  \item Планарность — возможность укладки графа на плоскости без пересечений рёбер.
  \item Любой граф можно вложить в трёхмерное пространство без пересечений.
\end{itemize}

\subsection*{Эйлерова характеристика}

Для связного планарного графа справедливо соотношение Эйлера:


\[
p - q + f = 2
\]


где:
\begin{itemize}
  \item $p$ — число вершин;
  \item $q$ — число рёбер;
  \item $f$ — число граней (областей, на которые разбивается плоскость).
\end{itemize}

Это соотношение используется для проверки планарности и анализа топологических свойств графа.

\subsection*{Теорема о пяти красках}

\textbf{Теорема:} Всякий планарный граф можно раскрасить не более чем в 5 цветов так, чтобы никакие две смежные вершины не имели одинаковый цвет.

\textbf{Дополнение:}
\begin{itemize}
  \item Теорема о четырёх красках утверждает, что достаточно 4 цветов — доказана с помощью компьютерной проверки.
  \item Теорема о пяти красках имеет классическое доказательство и используется как надёжная оценка сверху.
\end{itemize}

\subsection*{Применения}

\begin{itemize}
  \item Раскраска карт: соседние регионы должны иметь разные цвета.
  \item Составление расписаний: конфликты моделируются рёбрами.
  \item Планирование частот в сетях: соседние узлы не должны использовать одинаковые частоты.
\end{itemize}


    \newpage
}
