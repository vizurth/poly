{
	\section{Деревья сортировки. Ассоциативная память, способы реализации ассоциативной
памяти. Алгоритм поиска в дереве сортировки.}

\section*{Деревья сортировки}

В данном разделе рассматривается одно конкретное применение деревьев в программировании — \textbf{деревья сортировки} (также называемые \textbf{деревьями упорядочивания}).

Анализ охватывает как теоретические аспекты, например оценку высоты деревьев, так и практическую реализацию соответствующих алгоритмов.  
Кроме того, обсуждаются различные прагматические аспекты использования деревьев сортировки и затрагиваются некоторые смежные вопросы.

\section*{Ассоциативная память}

В практическом программировании часто используется механизм организации хранения и доступа к данным, называемый \textbf{ассоциативной памятью}.

При использовании ассоциативной памяти данные разбиваются на \textbf{записи}, каждая из которых ассоциирована с \textbf{ключом}.  
Ключ — это значение из упорядоченного множества, а записи могут иметь произвольную природу и различные размеры.  
Доступ к данным осуществляется по значению ключа, которое обычно выбирается простым, компактным и удобным для работы.

\subsection*{Примеры использования}

Ассоциативная память применяется во многих сферах:

\begin{itemize}
  \item \textbf{Словарь или энциклопедия:}  
  Ключ — заголовок статьи (обычно выделен жирным), запись — текст статьи.

  \item \textbf{Адресная книга:}  
  Ключ — имя контакта, запись — адресная информация (телефон, почтовый адрес и т.д.).

  \item \textbf{Банковские счета:}  
  Ключ — номер счёта, запись — финансовая информация (возможно, весьма сложная).
\end{itemize}

\subsection*{Основные операции}

Ассоциативная память должна поддерживать как минимум три базовые операции:

\begin{enumerate}
  \item \texttt{Add(key, record)} — добавление записи по ключу;
  \item \texttt{Find(key)} — поиск записи по ключу;
  \item \texttt{Delete(key)} — удаление записи по ключу.
\end{enumerate}

Эффективность каждой операции зависит от структуры данных, используемой для представления ассоциативной памяти.  
Общая эффективность определяется соотношением частот различных операций в конкретной программе.

\section*{Способы реализации ассоциативной памяти}

Для представления ассоциативной памяти используются следующие основные структуры данных:

\begin{enumerate}
  \item \textbf{Неупорядоченный массив}

  \item \textbf{Упорядоченный массив}  

  \item \textbf{Дерево сортировки}  
  Бинарное дерево, каждый узел которого содержит ключ и указатель на запись.  
  Свойство: значения ключей в узлах левого поддерева меньше, а в узлах правого — больше, чем в текущем узле.

  \item \textbf{Таблица расстановки (хэш-таблица)}  
  Использует хеш-функцию для быстрого доступа к записям по ключу.
\end{enumerate}

Основное внимание в этом разделе уделено алгоритмам выполнения операций с деревом сортировки.



    \newpage
}
