{
	\section{Потоки в сетях: определение потока, разрезы. Теорема Форда и Фалкерсона.
Алгоритм Форда-Фалкерсона. Коммуникационные сети.}

\section*{Сеть}

Пусть $G(V, E)$ — орграф с одним источником $s \in V$ и одним стоком $t \in V$, где $s \ne t$.  
\textbf{Сетью} $S = (G; c)$ называется орграф $G$ с заданной функцией $c : E \rightarrow \mathbb{R}_+$, сопоставляющей каждой дуге $e \in E$ неотрицательное действительное число $c(e)$, называемое \textbf{пропускной способностью}.

\section*{Поток}

\textbf{Потоком} в сети $S = (G; c)$, где $G(V, E)$, называется функция $f : E \rightarrow \mathbb{R}_+$, приписывающая каждой дуге $e$ неотрицательное число $f(e)$ — \textbf{поток по дуге} $e$, при выполнении следующих условий:
\begin{enumerate}
  \item $f(e) \leq c(e)$ — поток не превышает пропускную способность;
  \item Для любой вершины $v \in V \setminus \{s, t\}$ сумма входящих потоков равна сумме исходящих:
  

\[
  \sum_{(u, v) \in E} f(u, v) = \sum_{(v, w) \in E} f(v, w).
  \]


\end{enumerate}

\section*{Разрезы}

Пусть $S = (G; c)$ — сеть, где $G = (V, E)$.  
Если $X \subset V$, $s \in X$, $t \notin X$, и $Y = V \setminus X$, то множество


\[
R = \{ e \in E \mid e = (v, w),\ v \in X,\ w \in Y \}
\]


называется \textbf{разрезом} сети $S$ и обозначается $R = (X, Y)$.

\subsection*{Пропускная способность разреза}
Пропускной способностью разреза $R(X, Y)$ называется неотрицательное число


\[
c(R) = \sum_{e \in R} c(e).
\]

\section*{Теорема Форда–Фалкерсона}

\subsection*{Формулировка}
Пусть $S = (G; c)$ — сеть, где $G = (V, E)$.  
Величина максимального потока $p_{\text{max}}$ в сети $S$ совпадает с минимальной пропускной способностью $r_{\text{min}}$ её разрезов:


\[
p_{\text{max}} = r_{\text{min}}.
\]



\section*{Алгоритм Форда–Фалкерсона}

\subsection*{Трудоёмкость}


\[
\mathcal{O}(q \cdot p_{\text{max}})
\]



\subsection*{Шаги алгоритма}

\begin{enumerate}
  \item \textbf{Инициализация:}
  \begin{itemize}
    \item Для всех рёбер $(u, v)$ установить $f(u, v) := 0$.
    \item Задать исток $s$ и сток $t$.
  \end{itemize}

  \item \textbf{Поиск увеличивающего пути:}  
  Пока существует путь $p$ из $s$ в $t$ в остаточной сети (где $c_f(u, v) > 0$):
  \begin{enumerate}
    \item Найти минимальную остаточную пропускную способность:
    

\[
    c_f(p) := \min\{c_f(u, v) \mid (u, v) \in p\}.
    \]


    \item Для каждого ребра $(u, v) \in p$:
    \begin{itemize}
      \item Увеличить поток: $f(u, v) := f(u, v) + c_f(p)$.
      \item Уменьшить обратный поток: $f(v, u) := f(v, u) - c_f(p)$.
    \end{itemize}
  \end{enumerate}

  \item \textbf{Завершение:}  
  Когда увеличивающих путей больше нет, максимальный поток равен сумме потоков из $s$ в смежные вершины (или в $t$).
\end{enumerate}


\section*{Коммуникационные сети}

\begin{itemize}
  \item Моделью компьютерной сети может служить ориентированный граф, чьи узлы представляют компьютерные компоненты, а дуги — коммуникационные линии связи. Каждая дуга снабжена весом, обозначающим пропускную способность этой линии.

  \item Процедура \emph{статической маршрутизации} учитывает информацию о пропускной способности линии для определения фиксированного пути передачи между узлами. В целях оптимизации таких путей применяют алгоритм, близкий к алгоритму Дейкстры. Однако задержки могут возникать при сбоях и превышении пропускной способности сети.

  \item Процедура \emph{динамической маршрутизации} постоянно корректирует пропускную способность линий с учётом текущих потребностей. Набор правил или протокол позволяет узлам решать, когда и куда передавать новую информацию.

  \item Каждый узел поддерживает свою таблицу путей — задача оптимизации рассредоточена по всей сети.
\end{itemize}



    \newpage
}
