{
	\section{Выявление маршрутов с заданным количеством ребер. Определение
экстремальных путей на графах. Метод Шимбелла. Волновые алгоритмы. (составлял гпт)}

\subsection*{Маршруты с заданным количеством рёбер}

Пусть задан граф $G(V, E)$ и две вершины $s, t \in V$.  
Требуется определить, существует ли маршрут длины ровно $k$ рёбер.

\begin{itemize}
  \item Используется матрица смежности $A$ графа.
  \item Элемент $(A^k)_{st}$ показывает количество маршрутов длины $k$ между $s$ и $t$.
  \item Если $(A^k)_{st} > 0$, то существует хотя бы один маршрут длины $k$.
\end{itemize}

Таким образом, задача сводится к возведению матрицы смежности в степень.

\subsection*{Экстремальные пути}

Экстремальные пути — это пути, обладающие некоторым оптимальным свойством:

\begin{itemize}
  \item \textbf{Кратчайший путь} — минимальная суммарная длина.
  \item \textbf{Длиннейший простой путь} — максимальная длина без повторения вершин (NP-трудная задача).
  \item \textbf{Минимальный по числу рёбер путь}.
  \item \textbf{Максимальный потоковый путь} (в сетях).
\end{itemize}

Для кратчайших путей применяются алгоритмы Дейкстры, Беллмана–Форда, Флойда–Уоршелла.

\subsection*{Метод Шимбелла}

Метод Шимбелла — это итерационный метод динамического программирования для нахождения кратчайших путей.



\[
d^{(k)}(i,j) = \min \left( d^{(k-1)}(i,j),\ \min_{v \in V} \left( d^{(k-1)}(i,v) + w(v,j) \right) \right)
\]



\begin{itemize}
  \item $d^{(0)}(i,j)$ — исходная матрица весов.
  \item На каждом шаге учитываются пути длиной не более $k$ рёбер.
  \item После $n$ итераций получаем матрицу кратчайших расстояний.
\end{itemize}

Метод аналогичен алгоритму Флойда–Уоршелла, но формулируется как последовательное уточнение расстояний.

\subsection*{Волновые алгоритмы}

Волновые алгоритмы — это алгоритмы поиска в ширину (BFS-подобные), основанные на распространении «волны» по графу.

\begin{itemize}
  \item На первом шаге волна исходит из стартовой вершины.
  \item На каждом следующем шаге волна распространяется на все ещё не посещённые соседние вершины.
  \item Вершины помечаются номерами слоёв (расстоянием в рёбрах).
\end{itemize}

Применения:

\begin{itemize}
  \item Поиск кратчайшего пути в невзвешенном графе.
  \item Проверка связности.
  \item Построение уровневых графов (например, в алгоритме Диница).
\end{itemize}

Волновые алгоритмы являются основой для многих методов маршрутизации и анализа графов.


    \newpage
}
