{
	\section{Ядро графа. Алгоритм Магу. Спектры графов.}


\section*{32.1.
Ядро графа}

\textbf{Ядро графа} — это независимое доминирующее множество.

\begin{itemize}
  \item В полном графе каждая вершина является ядром.
  \item Любой граф имеет ядро.
\end{itemize}

\subsection*{Свойства ядра}

\begin{enumerate}
  \item Не может содержать петель или смежных вершин.
  \item Максимальное внутренне устойчивое подмножество.
  \item Одновременно максимально внутренне устойчивое и минимально внешне устойчивое.
  \item Граф без контуров всегда обладает ядром.
  \item Каждый орграф, не имеющий контуров нечётной длины, обладает ядром.
  \item Граф, не имеющий контуров нечётной длины, допускает ядро.
\end{enumerate}

\textbf{Независимое} множество — внутренне устойчивое.  
\textbf{Доминирующее} множество — внешне устойчивое.

\section*{32.2. 
Алгоритм Магу}

\subsection*{Для нахождения множества внутренней устойчивости (независимого)}

\begin{enumerate}
  \item Составляется матрица смежности.
  \item По таблице смежности выписываются парные дизъюнкции.
  \item Выражение приводится к ДНФ.
  \item Для любой элементарной конъюнкции выписываются недостающие элементы — они образуют множество внутренней устойчивости.
\end{enumerate}

\subsection*{Для нахождения множества внешней устойчивости (доминирующего)}

\begin{enumerate}
  \item Составляется матрица смежности.
  \item Дополняется единицами по главной диагонали.
  \item Для каждой строки выписываются дизъюнкции.
  \item Выражение приводится к ДНФ.
  \item Все вершины, входящие в элементарную конъюнкцию, образуют множество внешней устойчивости.
\end{enumerate}

\section*{Спектры графов}

\textbf{Спектр графа} — это множество собственных значений матрицы смежности графа.

Графы, не являющиеся изоморфными, но имеющие одинаковые характеристические многочлены, называются \textbf{коспектральными}.

Спектральные свойства графов находят применение в \textbf{квантовой химии}.


    \newpage
}
