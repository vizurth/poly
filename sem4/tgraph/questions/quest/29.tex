{
	\section{Задачи маршрутизации. VRP. Классификация. Методы решения задач VRP.}

	\section*{Задачи маршрутизации (VRP)}

Задачи маршрутизации актуальны в следующих областях:
\begin{enumerate}
  \item Сервисное обслуживание;
  \item Перевозка товаров;
  \item Общественный транспорт.
\end{enumerate}

Транспортная задача (Vehicle Routing Problem, VRP) формулируется на графе $G(V, E, C)$, где:
\begin{itemize}
  \item $V$ — множество вершин (точек доставки или клиентов);
  \item $E$ — множество рёбер (маршрутов между точками);
  \item $C$ — матрица стоимостей, определённая на множестве рёбер $E$.
\end{itemize}

Цель задачи VRP — минимизация общей стоимости при построении множества маршрутов, удовлетворяющих заданным ограничениям.

\section*{Классификация задач VRP}

\begin{itemize}
  \item Транспортная задача с учетом грузоподъемности (Capacitated VRP): транспортное средство имеет ограниченную грузоподъемность; Разновидностями задачи CVRP являются задачи 2L-CVRP и 3L-CVRP, в которых помимо грузоподъемности ТС учитывается размещение груза внутри ТС. В задаче 2L-CVRP — двумерное размещение, в задаче 3L-CVRP — трехмерное размещение.

  \item Транспортная задача с временными окнами (VRP with Time Windows, VRPTW): каждый клиент или заявка должны быть обслужены в определенный временной промежуток;

  \item Задача маршрутизации ТС с множеством депо (Multiple Depot Vehicle Routing Problem, MDVRP): имеется несколько депо для обслуживания клиентов.

  \item Задача маршрутизации ТС с раздельной доставкой (Split Delivery Vehicle Routing Problem, SDVRP): каждый клиент может быть обслужен одновременно несколькими ТС.

  \item Транспортная задача с пополнением и доставкой (VRP with Pick-up and Delivery, PDP): товары должны быть пополнены и доставлены в определенные точки;
\end{itemize}

\section*{Методы решения задач VRP}

\begin{enumerate}
  \item Перебор всех допустимых (пример: Метод ветвей и границ)
  \item Эвристические (пример: алгоритм заметания — группируем в кластеры и для каждого кластера решаем)
\end{enumerate}

\subsubsection*{Три вида эвристик:}
\begin{enumerate}
  \item Конструктивный (сначала для одной, потом для двух, потом т.д.)
  \item Двухфазный (группируем в кластеры)
  \item Улучшающие (берем любое и улучшаем)
\end{enumerate}


    \newpage
}
