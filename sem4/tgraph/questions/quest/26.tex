{
	\section{Эйлеровы циклы. Эйлеровы графы. Алгоритм Флери. Оценка числа эйлеровых
графов.}

\section*{Эйлеровы циклы. Эйлеровы графы}

Если граф содержит цикл (не обязательно простой), проходящий по всем рёбрам графа, то такой цикл называется \textbf{эйлеровым циклом}, а сам граф — \textbf{эйлеровым графом}.

\subsection*{Теорема}

Пусть граф $G$ связен и нетривиален. Тогда следующие утверждения эквивалентны:

\begin{enumerate}
  \item $G$ — эйлеров граф.
  \item Каждая вершина графа $G$ имеет чётную степень.
  \item Множество рёбер графа $G$ можно разбить на простые циклы.
\end{enumerate}

\section*{Алгоритм построения эйлерова цикла (Флери)}

Пусть граф $G$ — эйлеров, то есть связен и все вершины имеют чётную степень.  
Алгоритм Флери позволяет построить эйлеров цикл следующим образом:

\begin{enumerate}
  \item Инициализируем стек и помещаем в него произвольную вершину $v$.
  \item Пока стек не пуст:
  \begin{enumerate}
    \item Берём вершину $v$ с вершины стека.
    \item Если у $v$ нет смежных рёбер:
    \begin{itemize}
      \item Удаляем $v$ из стека и добавляем её в выходной массив (последовательность цикла).
    \end{itemize}
    \item Иначе:
    \begin{itemize}
      \item Выбираем смежную вершину $u$.
      \item Добавляем $u$ в стек.
      \item Удаляем ребро $(u, v)$ из графа.
    \end{itemize}
  \end{enumerate}
\end{enumerate}

В результате получаем последовательность вершин, задающую эйлеров цикл.

\section*{Оценка числа эйлеровых графов}

Обозначим:
\begin{itemize}
  \item $\mathcal{G}(p)$ — множество всех графов с $p$ вершинами;
  \item $\mathcal{E}(p)$ — множество эйлеровых графов с $p$ вершинами.
\end{itemize}

\subsection*{Теорема}

Эйлеровых графов \emph{почти нет}, то есть:


\[
\lim_{p \to \infty} \frac{|\mathcal{E}(p)|}{|\mathcal{G}(p)|} = 0
\]



Таким образом, при большом числе вершин доля эйлеровых графов среди всех возможных стремится к нулю.


    \newpage
}
