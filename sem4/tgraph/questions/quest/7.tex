{
	\section{Упорядочение дуг и вершин орграфа. Алгоритм Фалкерсона.}

	\subsection*{Упорядочивание дуг и вершин графа}

Под \textbf{упорядочиванием ациклического орграфа} понимается такое разбиение его вершин на группы, при котором:
\begin{enumerate}
  \item Вершины первой группы не имеют предшествующих, а последней — последующих.
  \item Вершины любой другой группы не имеют предшествующих в следующей группе.
  \item Вершины одной и той же группы дугами не соединяются.
\end{enumerate}

Такое разбиение всегда возможно.

\subsection*{Алгоритм Фалкерсона}

\begin{enumerate}
  \item Находим истоки — они образуют первую группу, нумеруем их в произвольном порядке.
  \item Вычёркиваем все пронумерованные вершины и дуги.
  \item Повторяем первый шаг для оставшегося графа — новая группа, новая нумерация.
  \item Продолжаем, пока не будут упорядочены все вершины.
\end{enumerate}

Аналогично можно упорядочить и дуги.

\subsection*{Матричный способ}

\begin{enumerate}
  \item Берём матрицу смежности.
  \item Находим столбцы, состоящие из нулей — это первая группа.
  \item Вычёркиваем найденные столбцы и соответствующие строки.
  \item Повторяем, пока не упорядочим все вершины.
\end{enumerate}


    \newpage
}
