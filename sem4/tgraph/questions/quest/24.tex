{
	\section{Кратчайший остов. Алгоритм построения остова экстремального веса. Алгоритм
Краскала. Алгоритм Прима. Алгоритм Борувки. Число остовов в связном
обыкновенном графе. Задача Штейнера}

\section*{Кратчайший остов}

\textbf{Остовный подграф} — это подграф, содержащий все вершины исходного графа.  
Остов задаётся множеством рёбер, поскольку вершины те же.  
Если рёбрам заданы длины (веса), то возникает задача нахождения \emph{кратчайшего остова} — остовного дерева минимального веса.

\section*{Алгоритмы построения кратчайшего остова}

Существуют следующие классические алгоритмы:

\begin{enumerate}
  \item \textbf{Алгоритм Краскала}  
  Сложность: $O(q \cdot \log q)$
  \begin{itemize}
    \item Берутся все вершины, но при этом множество рёбер берётся пустым.
    \item Рёбра графа сортируются по их весу.
    \item Выбирается ребро с минимальным весом и если его добавление не
вызовет появление цикла, то оно добавляется. Далее берётся
следующее ребро.
    \item  Повторяется пока есть рёбра не вставленные рёбра,
которые не вызовут появление цикла.
  \end{itemize}

  \item \textbf{Алгоритм Прима}  
  Сложность: $O(p^2)$ или $O(q + p \cdot \log p)$
  \begin{itemize}
    \item Выбираем произвольную вершину.
    \item Находим минимальное ребро, инцидентное этой вершине, и добавляем его в дерево.
    \item Далее выбираем минимальное ребро, соединяющее дерево с новой вершиной.
    \item Продолжаем, пока все вершины не включены в дерево.
  \end{itemize}

  \item \textbf{Алгоритм Борувки}  
  Сложность: $O(q \cdot \log p)$
  \begin{itemize}
	\item Каждая вершина представляется как отдельное дерево.
	\item Для каждого дерева находим самое дешёвое ребро, связывающее это
дерево с другим деревом.
	\item Соединяем деревья найденным ребром.
	\item Повторяем шаги 23 пока не останется только одно дерево.
  \end{itemize}
\end{enumerate}

\textit{Примечание:} Алгоритмы Прима и Краскала можно использовать также для нахождения \emph{максимального остова}, если изменить критерий выбора рёбер.

\section*{Число остовов в связном графе}

\textbf{Теорема Кирхгофа.}  
Число остовных деревьев в связном графе $G$ порядка $n \ge 2$ равно алгебраическому дополнению любого элемента матрицы Кирхгофа $B(G)$.

\subsection*{Матрица Кирхгофа}

Матрица Кирхгофа определяется как разность матрицы степеней и матрицы смежности:


\[
B(G) = 
\begin{pmatrix}
\deg(v_1) & 0 & \cdots & 0 \\
0 & \deg(v_2) & \cdots & 0 \\
\vdots & \vdots & \ddots & \vdots \\
0 & 0 & \cdots & \deg(v_n)
\end{pmatrix}
- A(G)
\]


где $A(G)$ — матрица смежности графа $G$.

\subsection*{Число остовов}

Число остовных деревьев графа $G$ равно любому \emph{алгебраическому дополнению} элемента матрицы $B(G)$, то есть:


\[
t(G) = \text{cof}_{ij}(B(G)),\quad \text{для любых } i, j \in \{1, \ldots, n\}
\]

\section*{Задача Штейнера}

\subsection*{Формулировка задачи}

Пусть задано множество городов на плоскости. Требуется определить минимальный набор железнодорожных линий (по суммарной длине), который обеспечит возможность переезда из любого города в любой другой.

\textbf{Важно:} кратчайший остов полного графа расстояний между городами не является решением задачи Штейнера, поскольку допускается введение дополнительных промежуточных точек.

\subsection*{Евклидова задача Штейнера}

Пусть задано произвольное множество $U$ точек на евклидовой плоскости.  
Необходимо соединить их непрерывными линиями так, чтобы любые две точки были связаны либо напрямую, либо через другие точки и соединяющие их линии, а суммарная длина всех линий была минимальной.

Допускается введение дополнительных точек (вершин), не входящих в исходное множество $U$, что отличает задачу Штейнера от задачи построения остовного дерева.

\subsection*{Графовая формулировка}

Задача Штейнера эквивалентна задаче нахождения остова минимального веса в порождённых подграфах графа $G$, множество вершин которых содержит $U$.

\subsection*{Сложность}

Задача Штейнера является NP-трудной и до конца не решена в общем случае.  
Для её решения применяются эвристики, аппроксимации и специальные алгоритмы для ограниченных классов графов.




    \newpage
}
