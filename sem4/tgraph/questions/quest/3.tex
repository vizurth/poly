{
	\section{Понятие изоморфизма. Изоморфизм графов (+2 теоремы). Инварианты графа.}

	\subsection*{Гомоморфизм}

Пусть $\mathcal{A} = \langle A; \varphi_1, \dots, \varphi_m \rangle$ и $\mathcal{B} = \langle B; \psi_1, \dots, \psi_m \rangle$ — две алгебры одного типа (одинаковые векторы арностей). Если существует функция $f : A \to B$, такая, что


\[
\forall i \in \{1, \dots, m\}:\quad f(\varphi_i(a_1, \dots, a_n)) = \psi_i(f(a_1), \dots, f(a_n)),
\]


то говорят, что $f$ — \textbf{гомоморфизм} из $\mathcal{A}$ в $\mathcal{B}$.

Действие гомоморфизма можно изобразить с помощью диаграммы:



\begin{figure}[h]
\centering
\begin{tikzcd}[row sep=large, column sep=large]
A \arrow[r, "\varphi"] \arrow[d, "f"'] & A \arrow[d, "f"] \\
B \arrow[r, "\psi"'] & B
\end{tikzcd}
\caption{Коммутативная диаграмма: $f \circ \varphi = \psi \circ f$}
\end{figure}




Диаграмма называется \textbf{коммутативной}, потому что условие гомоморфизма можно переписать с помощью суперпозиции функций:


\[
f \circ \varphi = \psi \circ f.
\]

\subsection*{Изоморфизм}

Пусть $\mathcal{A} = \langle A; \varphi_1, \dots, \varphi_m \rangle$ и $\mathcal{B} = \langle B; \psi_1, \dots, \psi_m \rangle$ — две алгебры одного типа, и $f : A \to B$ — изоморфизм или гоморфизм с биекцией. Тогда алгебры $\mathcal{A}$ и $\mathcal{B}$ изоморфны:


\[
\mathcal{A}^f \sim \mathcal{B}.
\]



\paragraph{Теорема 1.} Если $f : A \to B$ — изоморфизм, то $f^{-1} : B \to A$ тоже является изоморфизмом.

\paragraph{Теорема 2.} Отношение изоморфизма на множестве однотипных алгебр является эквивалентностью.

\subsection*{Изоморфизм графов}

Говорят, что два графа $G_1(V_1, E_1)$ и $G_2(V_2, E_2)$ \textbf{изоморфны} (обозначается $G_1 \sim G_2$ или $G_1 = G_2$), если существует биекция $h : V_1 \to V_2$, сохраняющая смежность:


\[
e_1 = (u, v) \in E_1 \iff e_2 = (h(u), h(v)) \in E_2.
\]

\subsection*{Теорема}
Изоморфизм графов есть отношение эквивалентности

\subsection*{Теорема 1.}
Графы изоморфны тогда и только тогда, когда их матрицы смежности вершин получаются друг из друга одновременными перестановками строк и столбцов.

\subsection*{Теорема 2.}
Графы (орграфы) изоморфны тогда и только тогда, когда их матрицы инцидентности получаются друг из друга произвольными перестановками строк и столбцов.

\subsection*{Инварианты}
Ну найдете сами


    \newpage
}
