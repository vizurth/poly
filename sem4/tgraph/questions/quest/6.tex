{
	\section{Операции над графами: локальные, алгебраические.}

	\subsection*{7.3.4. Операции над графами (продолжение)}

\begin{enumerate}
  \setcounter{enumi}{3}

  \item \textbf{Удаление вершины} $v$ из графа $G_1(V_1, E_1)$ (обозначение: $G_1(V_1, E_1) - v$, при условии $v \in V_1$) даёт граф $G_2(V_2, E_2)$, где
  

\[
  V_2 = V_1 \setminus \{v\}, \quad
  E_2 = E_1 \setminus \{e = (v_1, v_2) \mid v_1 = v \lor v_2 = v\}.
  \]


  Пример: $C_3 - v = K_2$.

  \item \textbf{Удаление ребра} $e$ из графа $G_1(V_1, E_1)$ (обозначение: $G_1(V_1, E_1) - e$, при условии $e \in E_1$) даёт граф $G_2(V_2, E_2)$, где
  

\[
  V_2 = V_1, \quad E_2 = E_1 \setminus \{e\}.
  \]


  Пример: $K_2 - e = K_2$.

  \item \textbf{Добавление вершины} $v$ в граф $G_1(V_1, E_1)$ (обозначение: $G_1(V_1, E_1) + v$, при условии $v \notin V_1$) даёт граф $G_2(V_2, E_2)$, где
  

\[
  V_2 = V_1 \cup \{v\}, \quad E_2 = E_1.
  \]


  Пример: $K_2 + v = K_2 \cup K_1$.

  \item \textbf{Добавление ребра} $e$ в граф $G_1(V_1, E_1)$ (обозначение: $G_1(V_1, E_1) + e$, при условии $e \notin E_1$) даёт граф $G_2(V_2, E_2)$, где
  

\[
  V_2 = V_1, \quad E_2 = E_1 \cup \{e\}.
  \]



  \item \textbf{Стягивание (правильного) подграфа} $A$ графа $G_1(V_1, E_1)$ (обозначение: $G_1(V_1, E_1)/A$, при условии $A \subset V_1$, $v \notin V_1$) даёт граф $G_2(V_2, E_2)$, где
  

\[
  V_2 = (V_1 \setminus A) \cup \{v\},
  \]


  

\[
  E_2 = E_1 \setminus \{e = (u, w) \mid u \in A \lor w \in A\} \cup \{e = (u, v) \mid u \in \Gamma(A) \setminus A\}.
  \]


  Пример: $K_4 / C_3 = K_2$.

  \item \textbf{Размножение вершины} $v$ графа $G_1(V_1, E_1)$ (обозначение: $G_1(V_1, E_1) \uparrow v$, при условии $v \in V_1$, $v' \notin V_1$) даёт граф $G_2(V_2, E_2)$, где



\[
V_2 = V_1 \cup \{v'\},
\]





\[
E_2 = E_1 \cup \{(v, v')\} \cup \{e = (u, v') \mid u \in \Gamma^+(v)\}.
\]



Пример: $K_2 \uparrow v = C_3$.

\end{enumerate}


    \newpage
}
